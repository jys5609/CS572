\section{Conclusion}
In this paper a new architecture for automatic lip-reading have been proposed.
The architecture consist of three new elements:
1) To use a bi-directional LSTM for the temporal model, 
2) The usage of batch-norm between the temporal model and the classification, and
3) data argumentation of a multi-view setup.
The architecture have been evaluated on both a single-view and multi-view setup.
In the multi-view setup different 4 different architecture have been evaluated.
For the proposed architecture a better performance is obtained, in comparison to the state-of-the-art architecture.
For single-view an average performance of 81.4\% is obtained.
This performance is 3.4\% better then the current state-of-the-art, of 77.9\%. 
The proposed architecture further have a better performance on each of the individual views, beside 90 degrees.
An architecture where batch-norm and bi-directional LSTM were used individually were also evaluated in this paper.
A decrease in the average performance were here observed when batch-norm were used alone.
For the bi-LSTM with no batch-norm an increasing performance were here observed.
The best performance could however be observed when the bi-LSTM and batch-norm were combined.
For the multi-view setup, the best performance were obtained by the use of a merge images architecture.
A performance of 90\% accuracy is here obtained.
This performance is 10\% better then the current state-of-the-art on 80\%.
Architectures where the batch-norm and bi-LSTM were used on its own where here also evaluated.
A better or equal performance where here observed, where the best performance gain where obtained with the combination of the two.
Augmentation of the training data where here also evaluated on each of the implementations.
The same or better performance where here obtained, with a significant performance increase for the combined architecture with batch-norm and bi-LSTM.
